\documentclass[UTF8,12pt]{article} % 12pt 为字号大小
\usepackage{amssymb,amsfonts,amsthm}
%\usepackage{fontspec,xltxtra,xunicode}
%\usepackage{times}
\usepackage{amsmath,bm}
\allowdisplaybreaks[4]
\usepackage{mdwlist}
\usepackage[colorlinks,linkcolor=blue]{hyperref}
\usepackage{cleveref}
\usepackage{float}
\usepackage{enumerate}
\usepackage{extarrows}

%----------
% 定义中文环境
%----------

\usepackage{xeCJK}
\setCJKmainfont[BoldFont={Heiti SC Light},ItalicFont={Kaiti SC Regular}]{Songti SC Regular}
\setCJKsansfont{Heiti SC Light}
\setCJKfamilyfont{song}{Songti SC Regular}
\setCJKfamilyfont{zhhei}{Heiti SC Light}
\setCJKfamilyfont{zhkai}{Kaiti SC Regular}
\setCJKfamilyfont{zhfs}{STFangsong}
\setCJKfamilyfont{zhli}{Libian SC Regular}
\setCJKfamilyfont{zhyou}{Yuanti SC Regular}

\newcommand*{\songti}{\CJKfamily{zhsong}} % 宋体
\newcommand*{\heiti}{\CJKfamily{zhhei}}   % 黑体
\newcommand*{\kaiti}{\CJKfamily{zhkai}}  % 楷体
\newcommand*{\fangsong}{\CJKfamily{zhfs}} % 仿宋
\newcommand*{\lishu}{\CJKfamily{zhli}}    % 隶书
\newcommand*{\yuanti}{\CJKfamily{zhyou}} % 圆体

%----------
% 版面设置
%----------
%首段缩进
\usepackage{indentfirst}
\setlength{\parindent}{2em}

%行距
\renewcommand{\baselinestretch}{1.2} % 1.2倍行距

%页边距
\usepackage[a4paper]{geometry}
\geometry{verbose,
  tmargin=2cm,% 上边距
  bmargin=2cm,% 下边距
  lmargin=2.5cm,% 左边距
  rmargin=2.5cm % 右边距
}


%----------
% 其他宏包
%----------
%图形相关
\usepackage[x11names]{xcolor} % must before tikz, x11names defines RoyalBlue3
\usepackage{graphicx}
\graphicspath{{figures/}}
\usepackage{pstricks,pst-plot,pst-eps}
\usepackage{subfig}
\def\pgfsysdriver{pgfsys-dvipdfmx.def} % put before tikz
\usepackage{tikz}

%原文照排
\usepackage{verbatim}

%网址
\usepackage{url}

%----------
% 定理、习题与解答环境
%----------
%定理环境
\usepackage[most]{tcolorbox}
\newtcbtheorem[number within=section]{theorem}{Theorem}{
     enhanced,
     breakable,
     sharp corners,
     attach boxed title to top left={
       yshifttext=-1mm
     },
     colback=blue!4!white,
     colframe=blue!75!black,
     fonttitle=\bfseries,
     boxed title style={
       sharp corners,
       size=small,
       colback=blue!75!black,
       colframe=blue!75!black,
     } 
}{theorem}

\newtcbtheorem[number within=section]{definition}{Definition}{
     enhanced,
     breakable,
     sharp corners,
     attach boxed title to top left={
       yshifttext=-1mm
     },
     colback=blue!4!white,
     colframe=blue!75!black,
     fonttitle=\bfseries,
     boxed title style={
       sharp corners,
       size=small,
       colback=blue!75!black,
       colframe=blue!75!black,
     } 
}{definition}

\newtcbtheorem[number within=section]{corollary}{Corollary}{
     enhanced,
     breakable,
     sharp corners,
     attach boxed title to top left={
       yshifttext=-1mm
     },
     colback=blue!4!white,
     colframe=blue!75!black,
     fonttitle=\bfseries,
     boxed title style={
       sharp corners,
       size=small,
       colback=blue!75!black,
       colframe=blue!75!black,
     } 
}{corollary}

\newtcbtheorem[number within=section]{myboxes}{Box}{
     enhanced,
     breakable,
     sharp corners,
     attach boxed title to top left={
       yshifttext=-1mm
     },
     %colback=white,
     colframe=black!75!white,
     fonttitle=\bfseries,
     boxed title style={
       sharp corners,
       size=small,
       colback=black!75!white,
       colframe=black!75!white,
     } 
}{myboxes}

%习题环境
\newtcbtheorem[]{exercise}{Problem}{
     enhanced,
     breakable,
     sharp corners,
     attach boxed title to top left={
       yshifttext=-1mm
     },
     colback=white,
     colframe=black,
     fonttitle=\bfseries,
     boxed title style={
       sharp corners,
       size=small,
       colback=black,
       colframe=black,
     } 
}{Problem}

%解答环境
\ifx\proof\undefined\
\newenvironment{proof}[1][\protect\proofname]{\par
\normalfont\topsep6\p@\@plus6\p@\relax
\trivlist
\itemindent\parindent
\item[\hskip\labelsep
\scshape
#1]\ignorespaces
}{%
\endtrivlist\@endpefalse
}
\fi

\renewcommand{\proofname}{\it{Solution}}

%==========
% 正文部分
%==========

\begin{document}

\title{Quiz 2}
%\author{Yuquan Chen}
\date{} % 若不需要自动插入日期,则去掉前面的注释;{ } 中也可以自定义日期格式
\maketitle

\begin{exercise}{20 points}{}
One dimensional infinity deep square well.\\
For a particle with mass $m$ in a potential 
$$V(x) = \begin{cases}+\infty & x<-\frac{a}{2 }\\ 0 & -\frac{a}{2}\le x\le \frac{a}{2} \\ +\infty & x>\frac{a}{2}\end{cases},$$
\begin{enumerate*}
\item (5 pts) write out the eigen energy $E_{n}$
\item (5 pts) write out the eigen wave function in position space $\phi_{n}(x)$
\item (10 pts) with $\psi(x,t=0) = \sqrt{\frac{1}{a}}\left(\cos\left(\frac{\pi x}{a}\right) - \sin\left(\frac{2\pi x}{a}\right)\right)$, derive and find $\psi(x, t = \frac{2ma^{2}}{\pi \hbar})$, hint: $\sin(x+y) = \sin(x)\cos(y) + \cos(x)\sin(y)$
\end{enumerate*}
\end{exercise}

\begin{proof}[Solution]~\par
(1) $E_{n} = \frac{n^{2}\pi^{2}\hbar^{2}}{2ma^{2}}$\par
(2) $\phi_{n}(x) = \sqrt{\frac{2}{a}}\sin\left(\frac{n\pi x}{a} + \frac{n\pi}{2}\right)$\par
(3) When $t = 0$,
\begin{align}
\psi(x,0) &= \sqrt{\frac{1}{a}}\cos\left(\frac{\pi x}{a}\right) - \sqrt{\frac{1}{a}}\sin\left(\frac{2\pi x}{a}\right) \\
&= \sqrt{\frac{1}{2}}\sqrt{\frac{2}{a}}\sin\left(\frac{\pi x}{a} + \frac{\pi}{2}\right) + \sqrt{\frac{1}{2}}\sqrt{\frac{2}{a}}\sin\left(\frac{2\pi x}{a} + \frac{2\pi}{2}\right) \\
&= \sqrt{\frac{1}{2}}\phi_{1} + \sqrt{\frac{1}{2}}\phi_{2}
\end{align}
so
\begin{align}
\psi(x,t) = \sqrt{\frac{1}{2}}e^{\frac{iE_{1}t}{\hbar}}\phi_{1} + \sqrt{\frac{1}{2}}e^{\frac{iE_{2}t}{\hbar}}\phi_{2}
\end{align}
where $E_{1} = \frac{\pi^{2}\hbar^{2}}{2ma^{2}},~ E_{2} = \frac{4\pi^{2}\hbar^{2}}{2ma^{2}},~ t = \frac{2ma^{2}}{\pi\hbar}$, so
\begin{align}
\psi(x,t) = -\sqrt{\frac{1}{2}}\phi_{1} + \sqrt{\frac{1}{2}}\phi_{2}
\end{align}
\end{proof}

\begin{exercise}{15 points}{}
One dimensional Harmonic Oscillator
$$H = \frac{\hat{p}^{2}}{2m} + \frac{1}{2}m\omega^{2}\hat{x}^{2}$$
with
$$a = \frac{1}{\sqrt{\hbar\omega}}\left(i\frac{\hat{p}}{\sqrt{2m}} + \sqrt{\frac{1}{2}m\omega^{2}}\hat{x}\right),~ a^{\dag} = \frac{1}{\sqrt{\hbar\omega}}\left(-i\frac{\hat{p}}{\sqrt{2m}} + \sqrt{\frac{1}{2}m\omega^{2}}\hat{x}\right)$$
\begin{enumerate*}
\item (2 pts) write out $[a,a^{\dag}]$
\item (1 pt) $a|n\rangle$
\item (1 pt) $a^{\dag}|n\rangle$
\item (1 pt) $E_{n}$
\item (10 pts) derive and evaluate $a^{3}(a^{\dag})^{2}|n=0\rangle$
\end{enumerate*}
\end{exercise}

\begin{proof}[Solution]~\par
(1) $[a,a^{\dag}] = 1$\par
(2) $a|n\rangle = \sqrt{n}|n-1\rangle$\par
(3) $a^{\dag}|n\rangle = \sqrt{n+1}|n+1\rangle$\par
(4) $E_{n} = \hbar\omega(n + \frac{1}{2})$\par
(5) $a^{3}a^{\dag}a^{\dag}|0\rangle = a^{3}a^{\dag}|1\rangle = a^{3}\sqrt{2}|2\rangle = 0$
\end{proof}

\begin{exercise}{15 points}{}
Dynamic of a spin-$\frac{1}{2}$ particle
$$H = \hbar\omega\sigma_{x},~ |\psi,t=0\rangle = \begin{pmatrix}1\\0\end{pmatrix}$$
\begin{enumerate*}
\item (5 pts) find $U = e^{-\frac{iHt}{\hbar}}$
\item (5 pts) find $|\psi,t\rangle$
\item (5 pts) evaluate $\langle\psi,t|\sigma_{z}|\psi,t\rangle$ to find the average value of measuring along $\sigma_{z}$ over time.
\end{enumerate*}
\end{exercise}

\begin{proof}[Solution]~\par
(1) The time evolution operator
\begin{align}
U(t) &= e^{\frac{-iHt}{\hbar}} = e^{-i\omega t\sigma_{x}} = \cos(\omega t)I - i\sin(\omega t)\sigma_{x} \\
&= \begin{pmatrix}\cos(\omega t) & -i\sin(\omega t)\\-i\sin(\omega t) & \cos(\omega t)\end{pmatrix}
\end{align}\par
(2) We have $|\psi,0\rangle = \begin{pmatrix}1\\0\end{pmatrix}$, then
\begin{align}
|\psi,t\rangle = U(t)|\psi,0\rangle = \begin{pmatrix}\cos(\omega t) & -i\sin(\omega t)\\-i\sin(\omega t) & \cos(\omega t)\end{pmatrix} \begin{pmatrix}1\\0\end{pmatrix} = \begin{pmatrix}\cos(\omega t)\\-i\sin(\omega t)\end{pmatrix}
\end{align}\par
(3) We know $\sigma_{z} = \begin{pmatrix}1 & 0\\0 & -1\end{pmatrix}$, so
\begin{align}
\langle\psi,t|\sigma_{z}|\psi,t\rangle &= \begin{pmatrix}\cos\omega t& i\sin\omega t\end{pmatrix}\begin{pmatrix}1 & 0\\0 & -1\end{pmatrix}\begin{pmatrix}\cos\omega t \\ -i\sin\omega t\end{pmatrix} \\
&= \cos^{2}\omega t - \sin^{2}\omega t = \cos(2\omega t)
\end{align}
\end{proof}


\end{document}