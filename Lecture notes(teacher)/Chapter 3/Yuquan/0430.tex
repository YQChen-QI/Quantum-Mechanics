\documentclass[UTF8,12pt]{article} % 12pt 为字号大小
\usepackage{amssymb,amsfonts,amsthm}
%\usepackage{fontspec,xltxtra,xunicode}
%\usepackage{times}
\usepackage{amsmath,bm}
\usepackage{mdwlist}
\usepackage[colorlinks,linkcolor=blue]{hyperref}
\usepackage{cleveref}
\usepackage{float}
\usepackage{enumerate}
\usepackage{extarrows}

%----------
% 定义中文环境
%----------

\usepackage{xeCJK}
\setCJKmainfont[BoldFont={Heiti SC Light},ItalicFont={Kaiti SC Regular}]{Songti SC Regular}
\setCJKsansfont{Heiti SC Light}
\setCJKfamilyfont{song}{Songti SC Regular}
\setCJKfamilyfont{zhhei}{Heiti SC Light}
\setCJKfamilyfont{zhkai}{Kaiti SC Regular}
\setCJKfamilyfont{zhfs}{STFangsong}
\setCJKfamilyfont{zhli}{Libian SC Regular}
\setCJKfamilyfont{zhyou}{Yuanti SC Regular}

\newcommand*{\songti}{\CJKfamily{zhsong}} % 宋体
\newcommand*{\heiti}{\CJKfamily{zhhei}}   % 黑体
\newcommand*{\kaiti}{\CJKfamily{zhkai}}  % 楷体
\newcommand*{\fangsong}{\CJKfamily{zhfs}} % 仿宋
\newcommand*{\lishu}{\CJKfamily{zhli}}    % 隶书
\newcommand*{\yuanti}{\CJKfamily{zhyou}} % 圆体

%----------
% 版面设置
%----------
%首段缩进
\usepackage{indentfirst}
\setlength{\parindent}{2em}

%行距
\renewcommand{\baselinestretch}{1.2} % 1.2倍行距

%页边距
\usepackage[a4paper]{geometry}
\geometry{verbose,
  tmargin=2cm,% 上边距
  bmargin=2cm,% 下边距
  lmargin=2.5cm,% 左边距
  rmargin=2.5cm % 右边距
}


%----------
% 其他宏包
%----------
%图形相关
\usepackage[x11names]{xcolor} % must before tikz, x11names defines RoyalBlue3
\usepackage{graphicx}
\graphicspath{{figures/}}
\usepackage{pstricks,pst-plot,pst-eps}
\usepackage{subfig}
\def\pgfsysdriver{pgfsys-dvipdfmx.def} % put before tikz
\usepackage{tikz}

%原文照排
\usepackage{verbatim}

%网址
\usepackage{url}

%----------
% 定理、习题与解答环境
%----------
%定理环境
\usepackage[most]{tcolorbox}
\newtcbtheorem[number within=section]{theorem}{Theorem}{
     enhanced,
     breakable,
     sharp corners,
     attach boxed title to top left={
       yshifttext=-1mm
     },
     colback=white,
     colframe=blue!75!black,
     fonttitle=\bfseries,
     boxed title style={
       sharp corners,
       size=small,
       colback=blue!75!black,
       colframe=blue!75!black,
     } 
}{theorem}

\newtcbtheorem[number within=section]{definition}{Definition}{
     enhanced,
     breakable,
     sharp corners,
     attach boxed title to top left={
       yshifttext=-1mm
     },
     colback=white,
     colframe=blue!75!black,
     fonttitle=\bfseries,
     boxed title style={
       sharp corners,
       size=small,
       colback=blue!75!black,
       colframe=blue!75!black,
     } 
}{definition}

\newtcbtheorem[number within=section]{corollary}{Corollary}{
     enhanced,
     breakable,
     sharp corners,
     attach boxed title to top left={
       yshifttext=-1mm
     },
     colback=white,
     colframe=blue!75!black,
     fonttitle=\bfseries,
     boxed title style={
       sharp corners,
       size=small,
       colback=blue!75!black,
       colframe=blue!75!black,
     } 
}{corollary}

\newtcbtheorem[number within=section]{myboxes}{Box}{
     enhanced,
     breakable,
     sharp corners,
     attach boxed title to top left={
       yshifttext=-1mm
     },
     %colback=white,
     colframe=black!75!white,
     fonttitle=\bfseries,
     boxed title style={
       sharp corners,
       size=small,
       colback=black!75!white,
       colframe=black!75!white,
     } 
}{myboxes}

%习题环境
\newtcbtheorem[number within=section]{exercise}{Problem}{
     enhanced,
     breakable,
     sharp corners,
     attach boxed title to top left={
       yshifttext=-1mm
     },
     colback=white,
     colframe=black,
     fonttitle=\bfseries,
     boxed title style={
       sharp corners,
       size=small,
       colback=black,
       colframe=black,
     } 
}{Problem}

%解答环境
\ifx\proof\undefined\
\newenvironment{proof}[1][\protect\proofname]{\par
\normalfont\topsep6\p@\@plus6\p@\relax
\trivlist
\itemindent\parindent
\item[\hskip\labelsep
\scshape
#1]\ignorespaces
}{%
\endtrivlist\@endpefalse
}
\fi

\renewcommand{\proofname}{\it{Solution}}

%==========
% 正文部分
%==========

\begin{document}

\title{Chapter 3}
\author{Yuquan Chen}
\date{2019/04/30} % 若不需要自动插入日期,则去掉前面的注释;{ } 中也可以自定义日期格式
\maketitle

\section{Coupled density matrix}

For a two spin-1/2 particle, $\rho_{A}$ for the first one, $\rho_{B}$ for the second. For example, 
\begin{align*}
\rho_{A} = |0\rangle\langle 0| = \begin{pmatrix}1&0\\0&0\end{pmatrix},~ \rho_{B} = \frac{1}{2} I = \frac{1}{2}\begin{pmatrix}1&0\\0&1\end{pmatrix}
\end{align*}
$$\Rightarrow \rho = \rho_{A} \otimes \rho_{B} = \begin{pmatrix}\frac{1}{2}&0&0&0\\0&\frac{1}{2}&0&0\\0&0&0&0\\0&0&0&0\end{pmatrix}$$
in general, for $\rho = \sum_{i}p_{i}|\psi_{i}\rangle\langle \psi_{i}|$ of one particle, the physical meaning is there is a statistics, that there is a probability $p_{i}$ for the particle at state $|\psi_{i}\rangle$. For 2 particles, the overall state is described as
\begin{align}
\rho &= \sum_{i}p_{i}|\psi_{i}^{(1)}\rangle\langle\psi_{i}^{(1)}| \otimes |\phi_{i}^{(2)}\rangle\langle\phi_{i}^{(2)}| \\
&= \sum_{i} p_{i}\left(|\psi_{i}^{(1)}\rangle \otimes |\phi_{i}^{(2)}\rangle\right) \cdot \left(\langle\psi_{i}^{(1)}|\otimes\langle\phi_{i}^{(2)}|\right)
\end{align}

\begin{myboxes}{Examples}{}
\textbf{Pure state case:} particle 1 at $|0\rangle$, particle 2 at $|1\rangle$. Density matrix $\rho = |01\rangle\langle 01|$

\textbf{Mixed state case:} For a two particle system, we have $\frac{1}{3}$ of chance two particles at $|\psi_{1}\rangle$, $\frac{1}{3}$ of chance two particles at $|\psi_{2}\rangle$, and $\frac{1}{3}$ of chance two particles at $|\psi_{3}\rangle$. The density matrix
$$\rho = \frac{1}{3}|\psi_{1}\rangle\langle\psi_{1}| + \frac{1}{3}|\psi_{2}\rangle\langle\psi_{2}| + \frac{1}{3}|\psi_{3}\rangle\langle\psi_{3}|$$
\end{myboxes}

\section{Dynamics}

For 1 particle, $\rho = \sum_{i} p_{i} |\psi_{i}(t)\rangle\langle\psi_{i}(t)|$, the Shr\"{o}dinger's equation
\begin{align}
i\hbar\frac{\partial}{\partial t}|\psi_{i}(t)\rangle = H|\psi_{i}(t)\rangle
\end{align}
so
\begin{align}
\dot{\rho} = \frac{d}{dt} \rho = \sum_{i} p_{i}\left(\frac{d}{dt}\right)
\end{align}
for multi particle system, we have $\rho_{\text{multi}}, H_{\text{multi}}$
\begin{align}
\Rightarrow \dot{\rho}_{\text{multi}} = \frac{1}{i\hbar}[H_{\text{multi}},\rho_{\text{multi}}]
\end{align}

\section{Trace}

Under basis $\{|\psi_{i}\rangle\}$,
\begin{align}
tr(\rho) = \sum_{i}\langle\psi_{i}|\rho|\psi_{i}\rangle
\end{align}
for multi particles, we need a $\{|\psi_{i}\rangle_{\text{multi}}\}$ as a basis,

\section{Partial trace}

\begin{definition}{Partial trace}{}
Suppose we have a two particle system, then the partial trace on particle A is
\begin{align}
tr_{A}(\rho) = \sum_{i} \left(\langle\psi_{i}|^{A} \otimes I^{B}\right) \cdot \rho \cdot \left(|\psi_{i}\rangle^{A} \otimes I^{B}\right)
\end{align}
\end{definition}

\subsection{Partial trace and entangled state}

If we have $\rho = |\psi\rangle\langle\psi|$, where $|\psi\rangle = \frac{1}{\sqrt{2}}(|00\rangle + |11\rangle)$ is an entangled state, then what is $tr_{A}(\rho)$? First, we can calculate the density matrix as follows:
\begin{align}
\rho &= |\psi\rangle\langle\psi| = \frac{1}{2}(|00\rangle + |11\rangle)(\langle 00| + \langle 11|) \\
&= \frac{1}{2}(|00\rangle\langle00| + |00\rangle\langle11| + |11\rangle\langle00| + |11\rangle\langle11|)
\end{align}
\begin{align}
\Rightarrow tr_{A}(\rho) = (\langle0|\otimes I) \cdot 
\end{align}











\end{document}