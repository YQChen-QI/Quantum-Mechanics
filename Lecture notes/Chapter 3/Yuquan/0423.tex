\documentclass[UTF8,12pt]{article} % 12pt 为字号大小
\usepackage{amssymb,amsfonts,amsthm}
%\usepackage{fontspec,xltxtra,xunicode}
%\usepackage{times}
\usepackage{amsmath,bm}
\usepackage{mdwlist}
\usepackage[colorlinks,linkcolor=blue]{hyperref}
\usepackage{cleveref}
\usepackage{float}
\usepackage{enumerate}

%----------
% 定义中文环境
%----------

\usepackage{xeCJK}
\setCJKmainfont[BoldFont={Heiti SC Light},ItalicFont={Kaiti SC Regular}]{Songti SC Regular}
\setCJKsansfont{Heiti SC Light}
\setCJKfamilyfont{song}{Songti SC Regular}
\setCJKfamilyfont{zhhei}{Heiti SC Light}
\setCJKfamilyfont{zhkai}{Kaiti SC Regular}
\setCJKfamilyfont{zhfs}{STFangsong}
\setCJKfamilyfont{zhli}{Libian SC Regular}
\setCJKfamilyfont{zhyou}{Yuanti SC Regular}

\newcommand*{\songti}{\CJKfamily{zhsong}} % 宋体
\newcommand*{\heiti}{\CJKfamily{zhhei}}   % 黑体
\newcommand*{\kaiti}{\CJKfamily{zhkai}}  % 楷体
\newcommand*{\fangsong}{\CJKfamily{zhfs}} % 仿宋
\newcommand*{\lishu}{\CJKfamily{zhli}}    % 隶书
\newcommand*{\yuanti}{\CJKfamily{zhyou}} % 圆体


%----------
% 版面设置
%----------
%首段缩进
\usepackage{indentfirst}
\setlength{\parindent}{2em}

%行距
\renewcommand{\baselinestretch}{1.2} % 1.2倍行距

%页边距
\usepackage[a4paper]{geometry}
\geometry{verbose,
  tmargin=2cm,% 上边距
  bmargin=2cm,% 下边距
  lmargin=2.5cm,% 左边距
  rmargin=2.5cm % 右边距
}


%----------
% 其他宏包
%----------
%图形相关
\usepackage[x11names]{xcolor} % must before tikz, x11names defines RoyalBlue3
\usepackage{graphicx}
\graphicspath{{figures/}}
\usepackage{pstricks,pst-plot,pst-eps}
\usepackage{subfig}
\def\pgfsysdriver{pgfsys-dvipdfmx.def} % put before tikz
\usepackage{tikz}

%原文照排
\usepackage{verbatim}

%网址
\usepackage{url}

%----------
% 定理、习题与解答环境
%----------
%定理环境
\usepackage[most]{tcolorbox}
\newtcbtheorem[number within=section]{theorem}{Theorem}{
     enhanced,
     breakable,
     sharp corners,
     attach boxed title to top left={
       yshifttext=-1mm
     },
     colback=white,
     colframe=blue!75!black,
     fonttitle=\bfseries,
     boxed title style={
       sharp corners,
       size=small,
       colback=blue!75!black,
       colframe=blue!75!black,
     } 
}{theorem}

\newtcbtheorem[number within=section]{definition}{Definition}{
     enhanced,
     breakable,
     sharp corners,
     attach boxed title to top left={
       yshifttext=-1mm
     },
     colback=white,
     colframe=blue!75!black,
     fonttitle=\bfseries,
     boxed title style={
       sharp corners,
       size=small,
       colback=blue!75!black,
       colframe=blue!75!black,
     } 
}{definition}

\newtcbtheorem[number within=section]{corollary}{Corollary}{
     enhanced,
     breakable,
     sharp corners,
     attach boxed title to top left={
       yshifttext=-1mm
     },
     colback=white,
     colframe=blue!75!black,
     fonttitle=\bfseries,
     boxed title style={
       sharp corners,
       size=small,
       colback=blue!75!black,
       colframe=blue!75!black,
     } 
}{corollary}

\newtcbtheorem[number within=section]{myboxes}{Box}{
     enhanced,
     breakable,
     sharp corners,
     attach boxed title to top left={
       yshifttext=-1mm
     },
     %colback=white,
     colframe=black!75!white,
     fonttitle=\bfseries,
     boxed title style={
       sharp corners,
       size=small,
       colback=black!75!white,
       colframe=black!75!white,
     } 
}{myboxes}

%习题环境
\newtcbtheorem[number within=section]{exercise}{Problem}{
     enhanced,
     breakable,
     sharp corners,
     attach boxed title to top left={
       yshifttext=-1mm
     },
     colback=white,
     colframe=black,
     fonttitle=\bfseries,
     boxed title style={
       sharp corners,
       size=small,
       colback=black,
       colframe=black,
     } 
}{Problem}

%解答环境
\ifx\proof\undefined\
\newenvironment{proof}[1][\protect\proofname]{\par
\normalfont\topsep6\p@\@plus6\p@\relax
\trivlist
\itemindent\parindent
\item[\hskip\labelsep
\scshape
#1]\ignorespaces
}{%
\endtrivlist\@endpefalse
}
\fi

\renewcommand{\proofname}{\it{Solution}}

%==========
% 正文部分
%==========

\begin{document}

\title{Chapter 3}
\author{Yuquan Chen}
\date{2019/04/23} % 若不需要自动插入日期,则去掉前面的注释;{ } 中也可以自定义日期格式
\maketitle

\textbf{Attention:} Quiz 4/30, two pieces of A4 double sided material, problems covering chapter 2. Basic calculas and linear algebra tutorials at 2 p.m. this Saturday.\\

\section{Recap}

Question: How to write the time evolution form of a state under the basis of $\{Y_{lm}\}$?\par
If we have $|\psi(0)\rangle = \sum_i c_i|\alpha_i\rangle \Rightarrow |\psi(t)\rangle = \sum_i c_i e^{-iE_n t/\hbar}|\alpha_i\rangle$,  where $H|\alpha_i\rangle = E_{i}|\alpha_{i}\rangle$, then we also have
\begin{align}
\langle\theta,\varphi|\psi(0)\rangle = \sum_{l,m}c_{l,m}Y_{l}^{m}\Rightarrow \langle\theta,\varphi|\psi(t)\rangle = \sum_{l,m}c_{l,m}Y_{l}^{m}e^{-iE_{l,m}t/\hbar}
\end{align}

\begin{myboxes}{Orbital angular momentum}{}
\begin{enumerate}
\item $L^{2}, L_{x}, L_{y}, L_{z},~[L^{2},L_{i}] = 0,~[L_{k},L_{l} = i\epsilon_{klm} L_{m}]$
\item Eigenfunctions for $L^{2}$ and $L_{i}$
\item Spherical harmonics $Y_{l}^{m},~ l = 0,1,2,...,~ m = \pm l,\pm (l-1),...$
\begin{align}
Y_{l}^{m}(\theta,\varphi) = \langle\theta,\varphi|l,m\rangle\propto e^{im\varphi}
\end{align}
$$Y_{0}^{0} = \sqrt{\frac{1}{4\pi}}$$
$$Y_{1}^{0} = \sqrt{\frac{3}{4\pi}}\cos\theta,~ Y_{1}^{\pm 1} = \sqrt{\frac{3}{8\pi}}\sin\theta ~e^{\pm i\varphi}$$
$$Y_{2}^{0} = \sqrt{\frac{5}{16\pi}}(3\cos^{2}\theta-1),~Y_{2}^{\pm1} = \sqrt{\frac{15}{8\pi}}\sin\theta\cos\theta~e^{\pm i\varphi},~Y_{2}^{\pm2} = \sqrt{\frac{15}{32\pi}}\sin^{2}\theta~e^{\pm2i\varphi}$$
\end{enumerate}
\end{myboxes}~\\

\begin{myboxes}{General properties of angular momentum}{}
\begin{enumerate}
\item $J^{2}, J_{x}, J_{y}, J_{z},~ [J^{2},J_{i}] = 0,~ [J_{k},J_{l}] = i\epsilon_{klm}J_{m}$, eigenstate $|j,m\rangle$
\item $J_{+},J_{-},~ [J_{z},J_{+}] = \hbar J_{+},~ [J_{z},J_{-}] = -\hbar J_{-}$
\item $J^{2}|j,m\rangle = j(j+1)\hbar^{2}|j,m\rangle,~J_{z}|j,m\rangle = m\hbar|j,m\rangle$, where\\ $j: \text{integer or half integer},~ m: -j,-j+1,...,j-1,j$
\item $\begin{cases}J_{+}|j,m\rangle = \hbar\sqrt{j(j+1)-m(m+1)}|j,m+1\rangle,~ J_{+}|j,j\rangle = 0 \\ J_{-}|j,m\rangle = \hbar\sqrt{j(j+1)-m(m-1)}|j,m-1\rangle,~J_{-}|j,-j\rangle = 0\end{cases}$
\begin{proof}[Proof]
Let $J_{+}|j,m\rangle = c|j,m+1\rangle$, we can evaluate $J_{-}J_{+}$ as:
$$J_{-}J_{+} = (J_{x}-iJ_{y})(J_{x}+iJ_{y}) = J_{x}^{2} + J_{y}^{2} + i[j_{x},J_{y}] = J^{2} - J_{z}^{2} - \hbar J_{z}$$
we have $\langle j,m|J_{-}J_{+}|j,m\rangle = \left|c\right|^{2}$, so
\begin{align*}
&\Rightarrow |c|^{2} = \hbar^{2}j(j+1) - m^{2}\hbar^{2} - m\hbar^{2} = \left(j(j+1)-m(m+1)\right)\hbar^{2}\\
&\Rightarrow J_{+}|j,m\rangle = \hbar\sqrt{j(j+1)-m(m+1)}|j,m+1\rangle
\end{align*}
Similarly, $J_{-}|j,m\rangle = \hbar\sqrt{j(j+1)-m(m-1)}|j,m-1\rangle$
\end{proof}
\end{enumerate}
\end{myboxes}

\section{Coupled spin-1/2 system}

\subsection{States}
We have 2 particles, each with spin-1/2, use basis $\{|0\rangle,|1\rangle\}$ for each. Combinations: $\{|00\rangle,|01\rangle,|10\rangle,|11\rangle\}$, use outer product $\otimes$ to join the two qubits. For example,
\begin{align}
|\psi_1\rangle = \begin{pmatrix}a_1\\b_1\end{pmatrix},~|\psi_2\rangle = \begin{pmatrix}a_2\\b_2\end{pmatrix}\longmapsto |\psi_1\rangle\otimes|\psi_2\rangle = \begin{pmatrix}a_1a_2\\a_1b_2\\b_1a_2\\b_1b_2\end{pmatrix}
\end{align}
For two qubits system, a general state could be expressed as $|\psi\rangle = c_0|00\rangle + c_1|01\rangle + c_2|10\rangle + c_3|11\rangle$.

\subsection{Operators}
Operator $A$ and $B$ for the first and the second particle, we have the operator $A\otimes B$:
\begin{align}\label{outer}
\left(A\otimes B\right)\cdot \left(|\psi_1\rangle\otimes|\psi_2\rangle\right) = A|\psi_1\rangle\otimes B|\psi_2\rangle
\end{align}
In matrix form, we have 
\begin{align}
A = \begin{pmatrix}a_{11} & a_{12}\\a_{21} & a_{22}\end{pmatrix},~ B = \begin{pmatrix}b_{11}&b_{12}\\b_{21}&b_{22}\end{pmatrix}
\end{align}
\begin{align}
A\otimes B = \begin{pmatrix}a_{11}B&a_{12}B\\a_{21}B&a_{22}B\end{pmatrix}
\end{align}
We can proof (\ref{outer}) mathematically. Here's some properties of outer products on operators.
\begin{myboxes}{Properties of outer products on operators}{}
(1)~$\left(A\otimes B\right)\cdot\left(C\otimes D\right) = (AC)\otimes (BD)$\\
(2)~$A\otimes B + C\otimes D \ne AC + BD$ (wrong dimensions)
\end{myboxes}
\begin{proof}[Proof]
\begin{align}
(C\otimes D)|\psi_1\rangle|\psi_2\rangle = C|\psi_{1}\rangle\otimes D|\psi_{2}\rangle
\end{align}
then apply $A\otimes B$ on it,
\begin{align}
\left(A\otimes B\right)\cdot\left(C\otimes D\right)\cdot|\psi_1\rangle|\psi_2\rangle &= \left(A\otimes B\right)\cdot \left[C|\psi_{1}\rangle\otimes D|\psi_{2}\rangle\right]\\
&= (AC|\psi_{1}\rangle)\otimes(BD|\psi_{2}\rangle)\\
&= (AC\otimes BD)\cdot|\psi_{1}\rangle|\psi_{2}\rangle
\end{align}
\end{proof}

\subsection{Entanglement, quantum logic gates}
A separable state: $|\varphi\rangle = |a\rangle\otimes|b\rangle$, but if $|\psi\rangle = \frac{1}{\sqrt{2}}\left(|00\rangle + |11\rangle\right)$, we cannot make it into the form $|a\rangle\otimes|b\rangle$, then the two particles are in entanglement state. Physically, we can say there's special correlation between the two particles. Entanglement is one special form of superposition. We can use some examples to clarify the entanglement and the non-entanglement state.
\begin{itemize}
\item Example(non-entanglement): $|\psi\rangle = \frac{1}{\sqrt{2}}\left(|00\rangle + |01\rangle\right) = |0\rangle\otimes\frac{1}{\sqrt{2}}\left(|0\rangle + |1\rangle\right)$
\item Example(non-entanglement): $|\psi\rangle = \frac{1}{\sqrt{2}}\left(|00\rangle + |10\rangle\right) = \frac{1}{\sqrt{2}}\left(|0\rangle + |1\rangle\right)\otimes|0\rangle$
\item Example(entanglement): \\
$|\psi\rangle = c\left(|00\rangle + \frac{1}{\sqrt{2}}(|0\rangle + |1\rangle)\otimes|1\rangle\right) = c\left(|00\rangle + \frac{1}{\sqrt{2}}|01\rangle + \frac{1}{\sqrt{2}}|11\rangle\right) \ne |a\rangle\otimes|b\rangle$ is not so much entangled.
\end{itemize}
Some other important entanglement states
\begin{itemize}
\item Bell state $\frac{1}{\sqrt{2}}\left(|00\rangle \pm |11\rangle\right),~\frac{1}{\sqrt{2}}\left(|01\rangle \pm|10\rangle\right)$
\item W-state $|w\rangle = \frac{1}{\sqrt{n}}\left(|10...0\rangle + |010...0\rangle + ... + |00...10\rangle + |00...01\rangle\right)$ is the permutations of the required state.
\end{itemize}

\subsubsection{Properties of entangled states}

(1)~Measurement

Suppose we create a state $|\psi\rangle = \frac{1}{\sqrt{2}}\left(|00\rangle + |11\rangle\right)$, then separate the two particles, after measure the state of the first particle(such as $|0\rangle$), then we know the state of the second one immediately, no matter how far they separated.\\

(2)~Quantum correlation V.S. classical correlation

In this case, we need to distinguish the state $|\psi\rangle$ with classical case that suppose we have $2n$ pairs of particles in total, $n$ of them are at state $|00\rangle$ and the rest are at state $|11\rangle$, then if we measure both the quantum case and the classical case, the result should be the same: half of the times we get $|00\rangle$, another half at $|11\rangle$.

Suppose we have $|\psi\rangle = \frac{1}{\sqrt{2}}\left(|00\rangle + |11\rangle\right)$, we can do a \textit{half flip},
\begin{align}
|0\rangle\longmapsto|+\rangle = \frac{1}{\sqrt{2}}(|0\rangle+|1\rangle)\\
|1\rangle\longmapsto|-\rangle = \frac{1}{\sqrt{2}}(|0\rangle-|1\rangle)
\end{align}
then in the quantum case,
\begin{align}
|\psi\rangle\longmapsto \frac{1}{\sqrt{2}}\left(|++\rangle + |--\rangle\right) = \frac{1}{\sqrt{2}}(|00\rangle + |11\rangle)
\end{align}
while in the classical case, it should be:
\begin{align}
|00\rangle\longmapsto |++\rangle = \frac{1}{2}(|00\rangle + |01\rangle + |10\rangle + |11\rangle) \\
|11\rangle\longmapsto |++\rangle = \frac{1}{2}(|00\rangle - |01\rangle - |10\rangle + |11\rangle)
\end{align}
If we repeat measurement along $\sigma_{z}$ after the \textit{half flip}, then we can easily find the difference.

\subsubsection{Entangling operators}
The Controlled-Not gate(CNOT) of two qubits
\begin{align}
CNOT = \begin{pmatrix}1&0&0&0\\0&1&0&0\\0&0&0&1\\0&0&1&0\end{pmatrix}
\end{align}
the truth table of the CNOT gate is:
\begin{table}[H]\centering
\begin{tabular}{|c|c|}\hline
Before&After\\ \hline
$|0\rangle\otimes|0\rangle$ & $|0\rangle\otimes|0\rangle$\\ \hline
$|0\rangle\otimes|1\rangle$ & $|0\rangle\otimes|1\rangle$\\ \hline
$|1\rangle\otimes|0\rangle$ & $|1\rangle\otimes|1\rangle$\\ \hline
$|1\rangle\otimes|1\rangle$ & $|1\rangle\otimes|0\rangle$\\ \hline
\end{tabular}
\end{table}

Suppose the input state
\begin{align}
|\psi_{\text{in}}\rangle = \frac{1}{\sqrt{2}}(|0\rangle + |1\rangle)\otimes|0\rangle = \frac{1}{\sqrt{2}}\begin{pmatrix}1\\0\\1\\0\end{pmatrix}
\end{align}
after a CNOT gate, the output state should be:
\begin{align}
|\psi_{\text{out}}\rangle &= CNOT\cdot|\psi_{\text{in}}\rangle = \frac{1}{\sqrt{2}}\begin{pmatrix}1&0&0&0\\0&1&0&0\\0&0&0&1\\0&0&1&0\end{pmatrix}\begin{pmatrix}1\\0\\1\\0\end{pmatrix} = \frac{1}{\sqrt{2}}\begin{pmatrix}1\\0\\0\\1\end{pmatrix} \\
&= \frac{1}{\sqrt{2}}(|00\rangle + |11\rangle)
\end{align}
which is an entangled state!

We need high fidelity to implement CNOT. For example, if we have $99.9\%$ rate making CNOT to be successful, then we can only apply CNOT for roughly $O(1000)$ times.

\subsubsection{How to construct a CNOT operation}
Suppose we have $H_1 =\hbar\omega\sigma_z^A\sigma_x^B$, then the evolution $U = e^{-iH_{1}t/\hbar} = e^{-i\omega t\sigma_{z}^{A}\sigma_{x}^{B}}$
\begin{align}
\Rightarrow U = \cos(\omega t)I - i\sin(\omega t)\sigma_{z}^{A}\sigma_{x}^{B}
\end{align}
we have
\begin{align}
\begin{cases}H_{0} = \hbar\omega_{0}\sigma_{z}^{1}\\H_{2} = \hbar\omega_{2}\sigma_{x}^{2}\end{cases}
\end{align}
then
\begin{align}
\begin{cases}CNOT = e^{-i\frac{\pi}{4}} e^{-iH_{0}t_{0}} e^{-iH_{2}t_{2}} e^{-iH_{1}t_{1}} \\t_{0} = \frac{7\pi}{4\omega_{0}},~t_{2} = \frac{7\pi}{4\omega_{2}},~t_{1} = \frac{\pi}{4\omega_{1}}\end{cases}
\end{align}
we call this kind of implementation \textit{pulse sequence}.


\end{document}